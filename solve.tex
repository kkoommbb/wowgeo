1.1. Для решения этой задачи нужно вспомнить, что вообще такое центр вписанной окружности. 
Оказывается (проставить ссылку), это точка пересечения биссектрис. Отсюда можно сразу сделать вывод: 
$\angle I_A H A = \angle I_A H C = \angle I_B H C = \angle I_B H B = 45^\circ$. Тут уже легко можно посчитать $\angle I_A H I_B$:
$\angle I_A H I_B = \angle I_A H C + \angle I_B H C = 45^\circ + 45^\circ = 90^\circ$.

[картинка]

Конечно, нам сильно помогло то, что $CH$ -- высота, мы смогли точно сказать, что все углы при вершине $H$ равны $45^\circ$. 
Но на самом деле $CH$ мог быть произвольным отрезком -- а исследуемый угол всё равно остался бы прямым! 
Это совсем несложно доказать. Если выкинуть из картинки всё ненужное, останется вот что:

[картинка с двумя окружностями, вписанными в смежные углы. Биссектрисы нарисованы пунктиром, отмечены равные углы]

Теперь наша задача по сути звучит так: посчитать угол между биссектрисами смежных углов. 
Но ведь очевидно, что он равен половине развёрнутого угла, то есть $\frac{1}{2} 180^\circ = 90^\circ$!

[спойлер Очевидно потому что...] $2\angle 1 + 2\angle 2 = 180^\circ$, значит, $\angle I_B H I_A = \angle 1 + \angle 2 = \frac{1}{2} 180^\circ = 90^\circ$.

1.2. Если вы решили предыдущую задачу (или внимательно прочитали решение), то понимаете, что эта задача практически ничем не отличается.
Если решение всё ещё не приходит -- ещё раз прочитайте решение прошлой задачи!

[спойлер О, решили и хотите проверить ответ?] Вспомним, что центры вписанных окружностей лежат на биссектрисах. 
Значит, если выкинуть из картинки всё ненужное, то она выглядит так:

[две окружности, вписанные в прямой угол]

Нужно найти угол между биссектрисами двух углов, в сумме составляющих $90^\circ$! Конечно же, этот угол равен $45^\circ$.
Ведь $2\angle 1 + 2\angle 2 = 90^\circ$, значит, $\angle I_B С I_A = \angle 1 + \angle 2 = \frac{1}{2} 90^\circ = 45^\circ$.

2.1. Согласен, задача слишком простая :) Нужно вспомнить самый простой признак признак подобия -- по двум равным углам.

Раз треугольники подобны и соответственные углы найдены, можно легко записать отношение соответственных сторон:

2.2. Конечно, просить доказать теорему "именно таким способом" -- не совсем корректная просьба. 
Тем более, что вы, скорее всего, помните такое доказательство теоремы Пифагора:

[картинка]

Тяжело, зная одно доказательство, придумывать другое.

[спойлер история про Колмогорова] 
Например, академик Колмогоров в юности интересовался историей, и даже подумывал стать профессиональным историком.
Однако после семинара по истории ему сказали, что он привёл слишком мало доказательств какого-то факта. Колмогоров 
решил заниматься математикой -- всё же это наука, в которой факт считался доказанным при наличии хотя бы одного любого
правильного доказательства.

Но вот в данном случае доказательство очень изящно.

[картинка]

Совершенно очевидно, что $S_{\triangle ABC} = S_{\triangle ACH} + S_{\triangle BCH}$. 
Из прошлой задачи мы знаем, что треугольники $ABC$, $ACH$, $BCH$ подобны, и знаем коэффициент подобия.
А ещё мы знаем, что площади подобных треугольников относятся, как квадраты коэффициента подобия!
Давайте это запишем:

$\frac{S_{\triangle ACH}}{S_{\triangle ABC}} = \left(\frac{AC}{AB}\right)^2$
$\frac{S_{\triangle BCH}}{S_{\triangle ABC}} = \left(\frac{BC}{AB}\right)^2$

То есть мы можем выразить площади маленьких треугольничков через площадь большого! 
Давайте же подставим эти площади в исходное очевидное равенство:

$S_{\triangle ABC} = S_{\triangle ABC} \cdot \left(\frac{BC}{AB}\right)^2 + S_{\triangle ABC} \cdot \left(\frac{AC}{AB}\right)^2$

Сокращаем на $S_{\triangle ABC}$, домножаем на $AB^2$ -- и вот она, теорема Пифагора!

$AB^2 = AC^2 + BC^2$


[спойлер у теоремы Пифагора ещё много доказательств] 
Вальтер Литцманн даже написал книгу, в которой рассказывается 400 доказательств теоремы Пифагора. 
Правда, не все они сильно отличаются друг от друга.
Например, доказательство, довольно сильно похожее на школьное, когда-то придумал и опубликовал президент США.
Подробнее об этом можно прочитать в статье в журнале "Квантик" №1 (2012) "Президент доказывает теоремы".

2.3. Хм, то, что надо доказать, как-то очень похоже на теорему Пифагора! Вот только прямоугольного треугольника с катетами $r_A$ и $r_B$ и гипотенузой $r$ как-то не видно...

Мы не будем искать теорему Пифагора непонятно где, а применим её в самом очевидном месте:

$AB^2 = AC^2 + BC^2$

И как это может быть связано с тем, что надо доказать? Разве что мы будем знать, что $AB$, $AC$, $BC$ имеют какое-то отношение к $r$, $r_B$, $r_A$...

Но ведь они имеют к ним отношение! 
$AB$, $BC$, $AC$ -- соответственно гипотенузы подобных треугольников $ABC$, $BCH$, $ACH$, а $r$, $r_B$, $r_A$ -- это их радиусы вписанных окружностей!

Тут самое время вспомнить (или узнать) такой важный факт:

В подобных треугольниках отношение длин соответственных элементов равно коэффициенту подобия.

Этот факт интуитивно практически очевиден.

[картинка кто-то дует и раздувает один треугольник в другой]

Давайте вначале применим его, а потом обсудим строгое доказательство.
Вначале применим его для пары треугольников $ABC$, $BCH$:

$\frac{r_B}{r} = \frac{BC}{AB}$, $r_B = r \cdot \frac{BC}{AB}$

Теперь для пары $\triangle ABC$ и $\triangle ACH$:

$\frac{r_A}{r} = \frac{AC}{AB}$, $r_A = r \cdot \frac{AC}{AB}$

Получается, что 
$r_A^2 + r_B^2 = r^2 \cdot  \left(\frac{AC}{AB}\right) + r^2 \left(\frac{BC}{AB}\right) = r^2 \cdot \frac{AC^2 + BC^2}{AB^2} = r^2$.

[спойлер А теперь строгое доказательство...] 

3.1. Раз есть вписанная окружность, а нас что-то спрашивают про её центр, то неплохо бы провести радиусы к точкам касания:

[картинка]

Посмотрим на четырёхугольник $C B_1 I A_1$. Он похож на квадрат! Докажем это.

Для начала это прямоугольник -- углы $B_1$ и $A_1$ прямые (радиус к точке касания перепендикулярен касательной), угол $C$ прямой, значит, $\angle A_1 I B_1 = 360^\circ - 90^\circ - 90^\circ - 90^\circ = 90^\circ$, то есть все углы прямые.

Осталось доказать, что все стороны равны. Главное -- $A_1I = B_1I$ как радиусы! А если в прямоугольнике смежные стороны равны, то это точно квадрат.

В квадрате диагональ в $\sqrt{2}$ раз больше стороны. В нашем квадрате сторона равна $r$, значит, диагональ $AI = \sqrt{2}r$.

Попутно мы доказали, что $CA_1 = CB_1 = r$. Это весьма полезный факт, он пригодится нам в следующей задаче.

3.2. Этот факт любопытен сам по себе -- обычно радиус вписанной окружности приходится считать как-то сложно (чаще всего, по формуле $r = \frac{S}{p} = \frac{\sqrt{p(p-a)(p-b)(p-c)}}{p}$, где $S$ -- площадь, $p$ -- полупериметр), а в случае прямоугольного треугольника формула получаетс суперпростой!

Решение опирается на очень полезный факт:

[картинка]

Пусть $ABC$ -- произвольный треугольник, $A_1$, $B_1$, $C_1$ -- точки касания, $AB = c$, $AC = b$, $BC = a$, $p = \frac{a+b+c}{2}$ -- полупериметр. Тогда $AB_1 = AC_1 = p - a$, $BC_1 = BA_1 = p - b$, $CA_1 = CB_1 = p - c$.

[спойлер Доказательство -- очень простое, попробуйте вначале сами!] доказательство...

В нашей задаче от этого полезного факта нужно вот что: $CA_1 = p - c = \frac{a+b+c}{2} - c = \frac{a + b - c}{2}$. 

В задаче 3.1 вы доказали (или прочитали в решении), что $r = CA_1$. Ага, то есть нам не надо вычислять радиус вписанной окружности, а надо узнать $CA_1$. Ура -- мы же узнали только что, что $CA_1 = \frac{a + b - c}{2}$, то есть $r = \frac{a + b - c}{2}$. Qued erat demonstrandum, как говорится.

3.3. Что мы по сути доказали в задаче 3.1? Мы доказали, что в любом прямоугольном треугольнике расстояние от центра вписанной окружности до прямого угла в $\sqrt{2}$ раз больше радиуса вписанной окружности. Применим этот факт к $\triangle ACH$ -- получим, что $HI_A = \sqrt{2} r_A$. Применим этот факт к $\triangle BCH$ -- получим, что $HI_B = \sqrt{2} r_B$. А теперь сложим: 

$HI_A^2 + HI_A^2 = (\sqrt{2} r_A)^2 + (\sqrt{2} r_B)^2 = 2r^2 = (\sqrt{2} r)^2 = AI^2$

(Второе равенство -- это задача 2.3.)

На самом деле мы вполне могли обойтись без задачи 3.1, повторив все рассуждения из задачи 2.3. Действительно, $AI$, $HI_A$, $HI_B$ -- соответственные 
