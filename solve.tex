1.1. Для решения этой задачи нужно вспомнить, что вообще такое центр вписанной окружности. 
Оказывается (проставить ссылку), это точка пересечения биссектрис. Отсюда можно сразу сделать вывод: 
$\angle I_A H A = \angle I_A H C = \angle I_B H C = \angle I_B H B = 45^\circ$. Тут уже легко можно посчитать $\angle I_A H I_B$:
$\angle I_A H I_B = \angle I_A H C + \angle I_B H C = 45^\circ + 45^\circ = 90^\circ$.

[картинка]

Конечно, нам сильно помогло то, что $CH$ -- высота, мы смогли точно сказать, что все углы при вершине $H$ равны $45^\circ$. 
Но на самом деле $CH$ мог быть произвольным отрезком -- а исследуемый угол всё равно остался бы прямым! 
Это совсем несложно доказать. Если выкинуть из картинки всё ненужное, останется вот что:

[картинка с двумя окружностями, вписанными в смежные углы. Биссектрисы нарисованы пунктиром, отмечены равные углы]

Теперь наша задача по сути звучит так: посчитать угол между биссектрисами смежных углов. 
Но ведь очевидно, что он равен половине развёрнутого угла, то есть $\frac{1}{2} 180^\circ = 90^\circ$!

[спойлер Очевидно потому что...] $2\angle 1 + 2\angle 2 = 180^\circ$, значит, $\angle I_B H I_A = \angle 1 + \angle 2 = \frac{1}{2} 180^\circ = 90^\circ$.

1.2. Если вы решили предыдущую задачу (или внимательно прочитали решение), то понимаете, что эта задача практически ничем не отличается.
Если решение всё ещё не приходит -- ещё раз прочитайте решение прошлой задачи!

[спойлер О, решили и хотите проверить ответ?] Вспомним, что центры вписанных окружностей лежат на биссектрисах. 
Значит, если выкинуть из картинки всё ненужное, то она выглядит так:

[две окружности, вписанные в прямой угол]

Нужно найти угол между биссектрисами двух углов, в сумме составляющих $90^\circ$! Конечно же, этот угол равен $45^\circ$.
Ведь $2\angle 1 + 2\angle 2 = 90^\circ$, значит, $\angle I_B С I_A = \angle 1 + \angle 2 = \frac{1}{2} 90^\circ = 45^\circ$.

2.1. Согласен, задача слишком простая :) Нужно вспомнить самый простой признак признак подобия -- по двум равным углам.

Раз треугольники подобны и соответственные углы найдены, можно легко записать отношение соответственных сторон:

2.2. Конечно, просить доказать теорему "именно таким способом" -- не совсем корректная просьба. 
Тем более, что вы, скорее всего, помните такое доказательство теоремы Пифагора:

[картинка]

Тяжело, зная одно доказательство, придумывать другое.

[спойлер история про Колмогорова] 
Например, академик Колмогоров в юности интересовался историей, и даже подумывал стать профессиональным историком.
Однако после семинара по истории ему сказали, что он привёл слишком мало доказательств какого-то факта. Колмогоров 
решил заниматься математикой -- всё же это наука, в которой факт считался доказанным при наличии хотя бы одного любого
правильного доказательства.

Но вот в данном случае доказательство очень изящно.

[картинка]

Совершенно очевидно, что $S_{\triangle ABC} = S_{\triangle ACH} + S_{\triangle BCH}$. 
Из прошлой задачи мы знаем, что треугольники $ABC$, $ACH$, $BCH$ подобны, и знаем коэффициент подобия.
А ещё мы знаем, что площади подобных треугольников относятся, как квадраты коэффициента подобия!
Давайте это запишем:

$\frac{S_{\triangle ACH}}{S_{\triangle ABC}} = \left(\frac{AC}{AB}\right)^2$
$\frac{S_{\triangle BCH}}{S_{\triangle ABC}} = \left(\frac{BC}{AB}\right)^2$

То есть мы можем выразить площади маленьких треугольничков через площадь большого! 
Давайте же подставим эти площади в исходное очевидное равенство:

$S_{\triangle ABC} = S_{\triangle ABC} \cdot \left(\frac{BC}{AB}\right)^2 + S_{\triangle ABC} \cdot \left(\frac{AC}{AB}\right)^2$

Сокращаем на $S_{\triangle ABC}$, домножаем на $AB^2$ -- и вот она, теорема Пифагора!

$AB^2 = AC^2 + BC^2$


[спойлер у теоремы Пифагора ещё много доказательств] 
Вальтер Литцманн даже написал книгу, в которой рассказывается 400 доказательств теоремы Пифагора. 
Правда, не все они сильно отличаются друг от друга.
Например, доказательство, довольно сильно похожее на школьное, когда-то придумал и опубликовал президент США.
Подробнее об этом можно прочитать в статье в журнале "Квантик" №1 (2012) "Президент доказывает теоремы".

2.3. 
