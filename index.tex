\documentclass[12pt]{article}

\usepackage[utf8]{inputenc}
\usepackage[russian]{babel}

\usepackage[margin=2truecm]{geometry}

\usepackage{paralist}

\usepackage{framed}

\begin{document}

\begin{enumerate}[\bf 1.]
    
\item
Пусть $ABC$ "--- правильный треугольник с прямым углом $C$, $H$ "--- основание высоты, опущенной на гипотенузу, $I_B$ и $I_A$ "--- центры окружностей, вписанных в треугольники $HBC$ и $HAC$ соответственно.

\begin{enumerate}[1.]
    \item Докажите, что $\angle I_A H I_B = 90^\circ$
    \item Докажите, что $\angle I_A C I_B = 45^\circ$
\end{enumerate}

\item
Пусть $r_B$, $r_A$, $r$ "--- радиусы окружностей, вписанных в $\triangle HBC$, $\triangle HAC$, $ABC$ соответственно. 
\begin{enumerate}[1.]
    \item Докажите, что $\triangle ABC$, $ \triangle CBH$,  $\triangle ACH$ подобны. Чему равны коэффициенты подобия?
    \item Докажите теорему Пифагора, используя результат предыдущей задачи.
    \item Докажите, что $r^2 = r_A^2 + r_B^2$.
\end{enumerate}

\item
Пусть $I$ "--- центр вписанной в $\triangle ABC$ окружности, $AB = c$, $BC = a$, $AC = b$.
\begin{enumerate}[1.]
    \item Докажите, что $AI = \sqrt{2} r$.
    \item Докажите, что $r = \frac{a+b-c}{2}$
    \begin{leftbar}
    обратите внимание на точки касания, запишите равенство отрезков касательных, проведите радиусы к точке касания, найдите квадратик.
    \end{leftbar}
    
    \item
    Докажите, что $AI^2 = HI_A^2 + HI_B^2$.
    \begin{leftbar}
    вспомните результат задачи 2.3.
    \end{leftbar}
    
    \item
    Докажите, что $AI = I_AI_B$.
    \begin{leftbar}
    используйте предыдущую задачу и задачу 1.1.
    \end{leftbar}

\end{enumerate}


\item

\begin{enumerate}[1.]
    \item  Докажите, что $\triangle I_A I_B H$ подобен $\triangle ABC$.
    \begin{leftbar}
    вспомните задачу 1.1 и задачи 3 и 4.
    \end{leftbar}
    
    \item Найдите коэффициент подобия треугольников из предыдущей задачи.
    \item Если есть два подобных треугольника на плоскости, то иногда один из них можно <<повернуть>> так, чтобы соответственные стороны стали параллельны. На сколько градусов <<повернут>> $\triangle I_A I_B H$ относительно $\triangle ABC$, если $\angle ABC = \alpha$?
    \item Докажите, что угол между прямыми $I_AI_B$ и $BC$ составляет $45^\circ$.

            \medskip

    Результат последней задачи можно сформулировать очень красиво:
    \begin{quote}
        Прямая $I_AI_B$ отсекает от треугольника $ABC$ равнобедренный прямоугольный треугольник.
    \end{quote}
\end{enumerate}

\item
\begin{enumerate}
    \item
    Докажите, что $\angle I_A I I_B = 135^\circ$.
    \item
    Докажите, что $CI_A \perp BI_B$.
    \item
    Докажите, что $AI \perp I_AI_B$
    \begin{leftbar}
    Аналогично прошлой задаче докажите, что $CI_B \perp AI_A$, а потом воспользуйтесь тем, что высоты пересекаются в одной точке. 
    \end{leftbar}
\end{enumerate}

\item
Пусть $I_AI_B$ пересекается с $AC$ в точке $A'$, а с $BC$  в точке $B'$. В предыдущем пункте мы доказали, что $\triangle A'B'C$ "--- прямоугольный равнобедренный.
\begin{enumerate}
    \item 
    Докажите, что $B'C = A'C = HC$.
    \begin{leftbar}
    Рассмотрите треугольники $I_B H C$ и $I_B B' C$ "--- они равны!. 
    \end{leftbar}
    \item 
    Докажите, что площадь треугольника $A'B'C$ не превосходит половины площади треугольника $ABC$.
    \begin{leftbar}
    Используйте результат предыдущей задачи и выразите площади треугольников через катеты $AC$ и $BC$.
    \end{leftbar}
\end{enumerate}
\end{enumerate}

\end{document}
